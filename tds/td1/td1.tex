\documentclass[a4paper,12pt]{amsart}
%%%%%%%%%%%%%%%%%%%%%%%%%%%%%%%%%%%%%%%%%%%%%
\usepackage[top=21mm,left=28mm,right=21mm,bottom=21mm,nohead,nofoot]{geometry}
\usepackage[french]{babel}
\usepackage[latin1]{inputenc}
\usepackage{enumitem}


% -------------- Set page lengths
\parskip=1mm
\parindent=0mm
\renewcommand{\baselinestretch}{1.3}

% -------------- set enumerate style
\setenumerate{itemsep=1mm,topsep=1mm,parsep=1mm,partopsep=0pt}
%-- Redefine the first level
\renewcommand{\theenumi}{\arabic{enumi}}
\renewcommand{\labelenumi}{\theenumi.}
%-- Redefine the second level
\renewcommand{\theenumii}{\alph{enumii}}
\renewcommand{\labelenumii}{\theenumii)}

% -------------- theorems
\theoremstyle{definition}
\newtheorem{exo}{Exercice}

% -------------- standard notations
\def\N{{\mathbb N}}
\def\R{{\mathbb R}}
\def\C{{\mathbb C}}
\newcommand{\MN}{{\mathcal{M}}_n}

% -------------- set {a,...,b}
\newcommand{\ens}[2]{\lbrace #1,\,\ldots,\, #2 \rbrace}

% -------------- autres
\def\ds{\displaystyle}

% -------------- haut de la page
\pagestyle{empty}
\newcommand{\hautdepage}[1]{
\thispagestyle{empty}
\clearpage
\textsc{\large Licence 3$^{{\mathaccent"7012e}me}$ ann�e, option Math�matiques} \hfill \textsc{\large 2012-2013}\\
\textsc{M65, Analyse num�rique matricielle}

\rule[0.5ex]{\textwidth}{0.1mm}
\vskip 3mm
\begin{center}
		{\sc{\Large #1}}
\end{center}
\rule[0.5ex]{\textwidth}{0.1mm}
}

%%%%%%%%%%%%%%%%%%%%%%%%%%%%%%%%%%%%%%%%%%%%%%%%%
\begin{document}
%%%%%%%%%%%%%%%%%%%%%%%%%%%%%%%%%%%%%%%%%%%%%%%%%
\hautdepage{TD1: Quelques �l�ments d'alg�bre matricielle}


%-----------------------------------
\begin{exo}  (Trigonalisation\,/\,diagonalisation)
	
	Soit $n \in \N^*$ et $A \in \MN(\C)$.
	\begin{enumerate}
		\item �noncer le th�or�me de d�composition de Schur.
		
		\item  Montrer que $A$ est normale si et seulement si il existe une matrice unitaire $U$ et une matrice diagonale $D$ contenant les valeurs propres de $A$ telles que $A=UDU^*$. On pourra calculer les �l�ments diagonaux de $TT^*$ et $T^*T$ o� $T$ est la matrice triangulaire sup�rieure telle que $A=UTU^*$ avec $U$ unitaire.
		
		\item Montrer que si $A$ est hermitienne, elle est diagonalisable dans une base de vecteurs propres orthogonaux et que ses valeurs propres sont r�elles.
		
		\item Montrer que si $A$ est unitaire, elle est diagonalisable dans une base de vecteurs propres orthogonaux et que ses valeurs propres ont pour module 1.
	\end{enumerate}
\end{exo}


%-----------------------------------
\begin{exo} (Valeurs propres du laplacien)
	
	On veut tout d'abord d�terminer les couples $(\lambda, v)$, avec $\lambda\in\R$ et $v\in C^2([0,1],\R)$  non identiquement nulle, solutions de
	\begin{equation}\label{ep}
		\left\{
		\begin{array}{l}
			-v''(x)=\lambda v(x) \mbox{ sur } ]0,1[\\
			v(0)=v(1)=0
		\end{array}
		\right.
	\end{equation}
	\begin{enumerate}
		\item Montrer que si $(\lambda,v)$  satisfait \eqref{ep} alors n�cessairement $\lambda >0$ ({\em suggestion} : multiplier l'�quation par $v$ puis int�grer sur $[0,1]$). On posera alors $\lambda=\omega^2$ avec $\omega\neq 0$.
		
		\item Montrer que les solutions sont de la forme :
		$$
		\lambda=(k\pi)^2 , v(x)=C\sin(k\pi x) ,\  k\in \N^{\ast},\ C\in\R^{\ast}.
		$$
	\end{enumerate}
	
	Soit $N\geq 3$. On consid�re  $A\in\mathcal{M}_N(\R)$, la matrice de discr�tisation du laplacien 1D:
	$$
	A=\left(
	\begin{array}{ccccc}
		2  		& -1 		& 0 		& \cdots 	& 0 		\\
		-1 		& \ddots 	& \ddots 	& \ddots 	& \vdots 	\\
		0  		& \ddots 	& \ddots 	& \ddots 	& 0 		\\
		\vdots 	& \ddots 	& \ddots 	& \ddots 	& -1 		\\
		0 		& \cdots 	& 0 		& -1 		& 2 		\\
	\end{array}
	\right)
	$$
	\begin{enumerate}
		\setcounter{enumi}{2}
		\item Montrer que, pour tout $1\leq k\leq N$, le vecteur $V_k=\left(\sin\Bigl(k\pi\ds\frac{i}{N+1}\Bigl)\right)_{1\leq i\leq N}$ est un vecteur propre de $A.$
		
		\item Montrer que les valeurs propres de $A$ sont :
		$$
		\lambda_k = 4\,\left( \sin\left(\ds\frac{k\,\pi}{2(N+1)}\right) \right) ^2,\quad 1\leq k\leq N
		$$
		
		\item En d�duire le rayon spectral de $A$.
	\end{enumerate}
\end{exo}


%-----------------------------------
\begin{exo} (M�thode de Gauss)
	
	Soit
	$$
		A=\left(
			\begin{array}{rrrr}
				 2 &  1 &   0 &  4 \\ 
				 4 &  1 &  -2 &  8 \\ 
				-4 & -2 &   3 & -7 \\ 
				 0 &  3 & -12 & -1
			\end{array}
		\right),
		\quad
		b=\left(
			\begin{array}{r}
				2 \\ 2 \\ -9 \\ 2
			\end{array}
		\right)
		\mbox{ et }
		X=\left(
			\begin{array}{r}
				x \\ y \\ z \\ t
			\end{array}
		\right)
	$$
	\begin{enumerate}
		\item �crire le syst�me lin�aire $AX=b$, not� $({\mathcal S})$. On note $A^{(1)}=A$ et $b^{(1)}=b$.

		\item R�soudre par la m�thode de Gauss le syst�me lin�aire $({\mathcal S})$. Pour chaque �tape $k$ ($k\in\{1,2,3\}$) de l'�limination,
		\begin{enumerate}
			\item pr�ciser la matrice $A^{(k+1)}$ et le second membre $b^{(k+1)}$ tels que $({\mathcal S})$ s'�crit sous la forme $A^{(k+1)}X=b^{(k+1)}$,

			\item donner la matrice $M^{(k)}$ telle que $A^{(k+1)}=M^{(k)}A^{(k)}$.

			\item calculer $L^{(k)}=(M^{(k)})^{-1}$.
		\end{enumerate}

		\item Donner la matrice triangulaire inf�rieure $L$ avec des 1 sur la diagonale et la matrice triangulaire sup�rieure $U$ telles que $A=LU$. En d�duire la valeur de $\det(A)$.
	\end{enumerate}
\end{exo}


%-----------------------------------
\begin{exo}
	
	Soit $n \in \N^*$. On d�finit $E_{ij}\in\MN(\R)$ la matrice avec un $1$ dans la position $(i,j)$ et $0$ partout ailleurs. On d�finit ensuite $V_{ij}(\lambda )=I +  \lambda E_{ij}$ , $\lambda \in\R , \,  i>j$.
	\begin{enumerate}
		\item Exprimer le produit $E_{ij} \, E_{kl}$ en fonction de $E_{il}$.

		\item Soit $A\in\MN(\R)$. Quels sont les r�sultats des op�rations
		$V_{ij}(\lambda )A$ et   $AV_{ij}(\lambda ) $ ?

		\item Soit $\lambda,\lambda'\in\R$ et $i,j,k$ tels que $i>j$ et $k>j$. Quelle est la forme de la matrice $V_{ij}(\lambda )V_{kj}(\lambda ') $ ?

		\item En d�duire l'expression de $(V_{ij}(\lambda ))^{-1}$.

		\item Soit $B^{(j)}\in \MN(\R)$ d�finie par $B^{(j)}=\ds\sum_{i=j+1}^n b_{i}^{(j)}E_{ij}$ et $L^{(j)}=I+B^{(j)}$.
		\begin{enumerate}
			\item Montrer que $B^{(j)}B^{(k)}=0$ pour tout $j\leq k$.
			\item En d�duire l'expression de $L^{(j)}L^{(k)}$ pour tout $j\leq k$,  puis celle de  $(L^{(j)})^{-1}$.
		\end{enumerate}
	\end{enumerate}
\end{exo}

\end{document}